\documentclass[aspectratio=169]{beamer}
%
% Choose how your presentation looks.
%
% For more themes, color themes and font themes, see:
% http://deic.uab.es/~iblanes/beamer_gallery/index_by_theme.html
%

\usepackage[english]{babel}
\usepackage[utf8]{inputenc}
\usepackage[T1]{fontenc}
\usepackage{listings}
\usepackage{fontspec}
\newfontfamily\menlo{Menlo}

\usetheme{BeamFlow} 

% language display optimization

\lstset{language=C++,
    backgroundcolor=\color{blue!10!backcolour},
    commentstyle=\color{codegreen},
    keywordstyle=\color{magenta},
    numberstyle=\tiny\menlo\color{codegray},
    stringstyle=\color{codepurple},
    basicstyle=\footnotesize\menlo, 
    showstringspaces=false
}

\title[Beamer Template]{Beamer Template for Research Presenattion}
\subtitle{A Brief Example}
\author[Short Name]{Full Name}
\institute{Where are you from?}
\date{June 14, 2020}

\begin{document}

\begingroup
\makeatletter
\setlength{\hoffset}{-.5\beamer@sidebarwidth}
\setbeamertemplate{navigation symbols}{}
\makeatother
\begin{frame}[plain]
     \titlepage
\end{frame}
\endgroup

% Uncomment these lines for an automatically generated outline.
\begin{frame}{Outline}
  \tableofcontents
\end{frame}

\section{Getting Started}

\begin{frame}[fragile]{Code Block}

\begin{itemize}
  \item Getting Started with a code block!
\end{itemize}

\vskip 0.5cm

\begin{exampleblock}{Examples}
  Show an example of \emph{Cpp} (with \textbf{OpenMP}) listing code block.
  \begin{lstlisting}[firstnumber=1, label=glabels, xleftmargin=10pt] 
#include <iostream>

int main() {
  std::cout << "hello" << std::endl;

  int sum = 0;
#pragma omp parallel for reduction(+: sum)
  for (int i = 0; i < 10000; i++)
    sum += i;
}
  \end{lstlisting}
\end{exampleblock}

\end{frame}

\begin{frame}{Cloumns}

\begin{columns}
  \begin{column}{0.5\textwidth}
    \begin{block}{Examples}
      Some examples of commonly used commands and features are included.
    \end{block}
  \end{column}
  \begin{column}{0.45\textwidth}  %%<--- here
    \begin{itemize}
      \item Your introduction goes here!
      \item Your introduction goes here!
    \end{itemize}
  \end{column}
\end{columns}

\vskip 1cm

\begin{alertblock}{Examples 2}
  \setbeamercolor{item projected}{bg=red!80!black,fg=white}
  \begin{enumerate}
    \item Chairman of the American National Standards Institute 
    \item Served as president of the International Organization for Standardization from 2003 to 2005.
  \end{enumerate}
\end{alertblock}

\end{frame}

\section{Examples}

\begin{frame}{Examples}

  \begin{enumerate}
      \item Chairman of the American National Standards Institute 
      \item Served as president of the International Organization for Standardization from 2003 to 2005.
  \end{enumerate}
  
\end{frame}

\subsection{Tables and Figures}

\begin{frame}{Tables and Figures}

\begin{enumerate}
    \item Chairman of the American National Standards Institute 
    \item Served as president of the International Organization for Standardization from 2003 to 2005.
\end{enumerate}


\begin{table}
\centering
\begin{tabular}{l|r}
Item & Quantity \\\hline
Widgets & 42 \\
Gadgets & 13
\end{tabular}
\caption{\label{tab:widgets}An example table.}
\end{table}

\end{frame}

\subsection{Mathematics}

\begin{frame}{Readable Mathematics}

Let $X_1, X_2, \ldots, X_n$ be a sequence of independent and identically distributed random variables with $\text{E}[X_i] = \mu$ and $\text{Var}[X_i] = \sigma^2 < \infty$, and let
\[ S_n = \frac{X_1 + X_2 + \cdots + X_n}{n}
      = \frac{1}{n}\sum_{i}^{n} X_i \]
denote their mean. Then as $n$ approaches infinity, the random variables $\sqrt{n}(S_n - \mu)$ converge in distribution to a normal $\mathcal{N}(0, \sigma^2)$.

\end{frame}

\end{document}
